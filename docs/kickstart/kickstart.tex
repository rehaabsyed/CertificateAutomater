%%%%%%%%%%%%%%%%%%%%%%%%%%%%%%%%%%%%%%%%%
% Lachaise Assignment
% LaTeX Template
% Version 1.0 (26/6/2018)
%
% This template originates from:
% http://www.LaTeXTemplates.com
%
% Authors:
% Marion Lachaise & François Févotte
% Vel (vel@LaTeXTemplates.com)
%
% License:
% CC BY-NC-SA 3.0 (http://creativecommons.org/licenses/by-nc-sa/3.0/)
% 
%%%%%%%%%%%%%%%%%%%%%%%%%%%%%%%%%%%%%%%%%


%----------------------------------------------------------------------------------------
%    PACKAGES AND OTHER DOCUMENT CONFIGURATIONS
%----------------------------------------------------------------------------------------

\documentclass[11pt]{article}

\input{structure.tex} % Include the file specifying the document structure and custom commands

\pagestyle{fancy}
\fancyhf{}
\renewcommand{\headrulewidth}{2pt}
\renewcommand{\footrulewidth}{1pt}
\lhead{\textit{Certificate Maker Kickstart Manual}}
\rhead{\leftmark}
\cfoot{Page \thepage}

%----------------------------------------------------------------------------------------
%    ASSIGNMENT INFORMATION
%----------------------------------------------------------------------------------------

\title{Certificate Maker Kickstart Manual} % Title of the assignment

\author{Harrison Outram\\ Membership Officer}

\date{Curtin IET On Campus --- \today} % University, school and/or department name(s) and a date

%----------------------------------------------------------------------------------------

\begin{document}

%----------------------------------------------------------------------------------------
%    TITLE PAGE
%----------------------------------------------------------------------------------------

\begin{titlepage} % Suppresses displaying the page number on the title page and the subsequent page counts as page 1
    \newcommand{\HRule}{\rule{\linewidth}{0.5mm}} % Defines a new command for horizontal lines, change thickness here
    
    \center % Centre everything on the page
    
    %------------------------------------------------
    %    Headings
    %------------------------------------------------
    
    \textsc{\LARGE Curtin IET On Campus}\\[1.5cm] % Main heading such as the name of your university/college
    
    \textsc{\LARGE Institute of Engineering and Technology}\\[0.5cm] % Major heading such as course name
    
    %------------------------------------------------
    %    Title
    %------------------------------------------------
    
    \HRule\\[0.4cm]

    \vspace{0.4cm}
    
    {\Huge\bfseries Certificate Maker Kickstart Manual}\\[0.4cm] % Title of your document
    
    \HRule\\[1.5cm]
    
    %------------------------------------------------
    %    Author(s)
    %------------------------------------------------
    
    \begin{center}
        \large
        Harrison G. Outram \\
        Membership Officer
    \end{center}
    
    %------------------------------------------------
    %    Date
    %------------------------------------------------
    
    \vfill\vfill\vfill % Position the date 3/4 down the remaining page
    
    {\Large\today} % Date, change the \today to a set date if you want to be precise
    
    %------------------------------------------------
    %    Logo
    %------------------------------------------------
    
    \vfill
    \includegraphics[width=0.8\textwidth]{../../assets/IET_Logo_Blue_RGB.pdf}\\
     
    %----------------------------------------------------------------------------------------
    
    \vfill % Push the date up 1/4 of the remaining page
    
\end{titlepage}

\pagenumbering{roman}

\tableofcontents

\bigskip

\listoffigures

\newpage

%----------------------------------------------------------------------------------------
%    INTRODUCTION
%----------------------------------------------------------------------------------------

\pagenumbering{arabic}
\setcounter{page}{1}

\section{Introduction}

\subsection{Context}

Curtin IET On Campus (or ``CIET'' for short) provides multiple industry talks and workshops to STEM students (particularly engineering and computer related science). As part of CIET's commitment to quality, attendees shall recieve a certificate of attendance stating the name of the event, the club name, the president's full name with a signature, the the attendee's full name, and the number of approved CPD hours. This certificate of attendance must be done in a PDF document with vector graphics where possible, then emailed to attendees as soon as possible. This is particularly important for undergraduate engineering students, as one graduation requirement is to obtain a minimum of 16 weighted hours (or five and a third actual) in the PRES category (technical presentations and workshops by a professional body). By creating and sending out these certificates, CIET is holding itself to a high standard of attendee satisfaction and professional development.

\subsection{Problem}

Despite the necessity of creating and sending certificates, the process for doing so is extremely time consuming and error prone. As shown in Figure \ref{fig:old-solution}, this process is comprised of four stages: (1) the attendee record must be collected from the events team and checked for errors, e.g. multiple registrations from one person or missing but required information. (2) one certificate is made for each attendee via the template on CIET's Canva account. Each certificate is then checked one by one for errors, being recreated if erroneous. (3) the certificates are uploaded to CIET's MailChimp account, where MailChimp auto-generates a URI for each certificate, required later on. Unfortunately, these URIs involve a randomly generated hexadecimal string, meaning each URI must be recorded manually in the attendee record, then checked for errors. Once the attendee record has been updated with certificate URIs, it is uploaded to MailChimp. (4) The certificates are checked one last time by creating a mock campaign on MailChimp, going through each attendee, downloading the certificate, then checking the certificate for errors and fixing where erroneous. Despite every certificate being checked three times during this process, it is still possible for erroneous certificates to be sent out, which has happened on at least one occasion. Clearly, this process is extremely time consuming and error prone, leading to attendee frustration and a negative impact on CIET's reputation.

\newpage

\begin{figure}[h!]
    \includegraphics[width=\textwidth]{figures/old_solution.pdf}
    \caption{Old solution flowchart.}
    \label{fig:old-solution}
\end{figure}

\newpage

\section{Solution Overview}

\subsection{Rundown}



\subsection{Minimal Viable Product}



\subsection{Professional Product}



\subsection{Ideal Product}



\newpage

\section{Prerequisite Knowledge}

\subsection{Python Language}



\subsection{HTTP Requests and Responses}



\subsection{Application Programming Interface}



\newpage

\section{PDF Editing and Creation}



\newpage

\section{MailChimp Authorisation}

\subsection{Using API and Server Keys}




\subsection{Using OAuth 2.0}



\newpage

\section{Certificate Uploading}




\newpage

\section{Contact Updating}



\newpage

\section{User Interface}

\subsection{Command Line Interface}






\subsection{Desktop GUI}




\subsection{Web App}



\newpage

\section{Version Control}

\subsection{What is Git?}



\subsection{Using Github}



\subsection{Recording Library Requirements}



\subsection{Virtual Environments}



\newpage

\section{Coding Standards}

\subsection{Readability and Maintainability}



\subsection{Documentating using Sphinx}



\newpage

%%\printbibliography

\end{document}
%%%%%%%%%%%%%%%%%%%%%%%%%%%%%%%%%%%%%%%%%
% Lachaise Assignment
% LaTeX Template
% Version 1.0 (26/6/2018)
%
% This template originates from:
% http://www.LaTeXTemplates.com
%
% Authors:
% Marion Lachaise & François Févotte
% Vel (vel@LaTeXTemplates.com)
%
% License:
% CC BY-NC-SA 3.0 (http://creativecommons.org/licenses/by-nc-sa/3.0/)
% 
%%%%%%%%%%%%%%%%%%%%%%%%%%%%%%%%%%%%%%%%%


%----------------------------------------------------------------------------------------
%    PACKAGES AND OTHER DOCUMENT CONFIGURATIONS
%----------------------------------------------------------------------------------------

\documentclass[11pt]{article}

\input{structure.tex} % Include the file specifying the document structure and custom commands

\usepackage{draftwatermark}
\SetWatermarkText{\textsf{\textbf{CONFIDENTAL}}}
\SetWatermarkScale{2.5}
\SetWatermarkColor[gray]{0.8}
\SetWatermarkAngle{50}

\pagestyle{fancy}
\fancyhf{}
\renewcommand{\headrulewidth}{2pt}
\renewcommand{\footrulewidth}{1pt}
\lhead{\textit{Certificate Maker Kickstart Manual}}
\rhead{\leftmark}
\cfoot{Page \thepage}

%----------------------------------------------------------------------------------------
%    ASSIGNMENT INFORMATION
%----------------------------------------------------------------------------------------

\title{Certificate Maker Kickstart Manual} % Title of the assignment

\author{Harrison Outram\\ Membership Officer}

\date{Curtin IET On Campus --- \today} % University, school and/or department name(s) and a date

%----------------------------------------------------------------------------------------

\begin{document}

%----------------------------------------------------------------------------------------
%    TITLE PAGE
%----------------------------------------------------------------------------------------

\begin{titlepage} % Suppresses displaying the page number on the title page and the subsequent page counts as page 1
    \newcommand{\HRule}{\rule{\linewidth}{0.5mm}} % Defines a new command for horizontal lines, change thickness here
    
    \center % Centre everything on the page
    
    %------------------------------------------------
    %    Headings
    %------------------------------------------------
    
    \textsc{\LARGE Curtin IET On Campus}\\[1.5cm] % Main heading such as the name of your university/college
    
    \textsc{\LARGE Institute of Engineering and Technology}\\[0.5cm] % Major heading such as course name
    
    %------------------------------------------------
    %    Title
    %------------------------------------------------
    
    \HRule\\[0.4cm]

    \vspace{0.4cm}
    
    {\Huge\bfseries Certificate Maker Kickstart Manual}\\[0.4cm] % Title of your document
    
    \HRule\\[1.5cm]
    
    %------------------------------------------------
    %    Author(s)
    %------------------------------------------------
    
    \begin{center}
        \large
        Harrison G. Outram \\
        Membership Officer
    \end{center}
    
    %------------------------------------------------
    %    Date
    %------------------------------------------------
    
    \vfill\vfill\vfill % Position the date 3/4 down the remaining page
    
    {\Large\today} % Date, change the \today to a set date if you want to be precise
    
    %------------------------------------------------
    %    Logo
    %------------------------------------------------
    
    \vfill
    \includegraphics[width=0.8\textwidth]{../../assets/IET_Logo_Blue_RGB.pdf}\\
     
    %----------------------------------------------------------------------------------------
    
    \vfill % Push the date up 1/4 of the remaining page
    
\end{titlepage}

\pagenumbering{roman}

\tableofcontents

\bigskip

\listoffigures

\newpage

%----------------------------------------------------------------------------------------
%    INTRODUCTION
%----------------------------------------------------------------------------------------

\pagenumbering{arabic}
\setcounter{page}{1}

\section{Introduction}

\subsection{Context}

Curtin IET On Campus (or ``CIET'' for short) provides multiple industry talks and workshops to STEM students (particularly engineering and computer related science). As part of CIET's commitment to quality, attendees shall recieve a certificate of attendance stating the name of the event, the club name, the president's full name with a signature, the the attendee's full name, and the number of approved CPD hours. This certificate of attendance must be done in a PDF document with vector graphics where possible, then emailed to attendees as soon as possible. This is particularly important for undergraduate engineering students, as one graduation requirement is to obtain a minimum of 16 weighted hours (or five and a third actual) in the PRES category (technical presentations and workshops by a professional body). By creating and sending out these certificates, CIET is holding itself to a high standard of attendee satisfaction and professional development.

\subsection{Problem}

Despite the necessity of creating and sending certificates, the process for doing so is extremely time consuming and error prone. As shown in Figure \ref{fig:old-solution}, this process is comprised of four stages: (1) the attendee record must be collected from the events team and checked for errors, e.g. multiple registrations from one person or missing but required information. (2) one certificate is made for each attendee via the template on CIET's Canva account. Each certificate is then checked one by one for errors, being recreated if erroneous. (3) the certificates are uploaded to CIET's MailChimp account, where MailChimp auto-generates a URI for each certificate, required later on. Unfortunately, these URIs involve a randomly generated hexadecimal string, meaning each URI must be recorded manually in the attendee record, then checked for errors. Once the attendee record has been updated with certificate URIs, it is uploaded to MailChimp. (4) The certificates are checked one last time by creating a mock campaign on MailChimp, going through each attendee, downloading the certificate, then checking the certificate for errors and fixing where erroneous. All up, this process takes at least two hours per 50 attendees for someone who has gone through it before, and \textbf{far} longer for someone who has not. Worse yet, despite every certificate being checked three times during this process, it is still possible for erroneous certificates to be sent out, which has happened on at least one occasion. Clearly, this process is extremely time consuming and error prone, leading to attendee frustration and a negative impact on CIET's reputation.

\newpage

\begin{figure}[h!]
    \includegraphics[width=\textwidth]{figures/old_solution.pdf}
    \caption{Old solution flowchart.}
    \label{fig:old-solution}
\end{figure}

\newpage

\section{Solution Overview}

\subsection{Rundown}

After some research, it has been discovered that stages 2, 3, and 4 of the old solution can be automated. For stage 1, instead of logging into Canva and creating each certificate manually, the template can be downloaded once and used to automatically create every certificate. The MailChimp API can be used to automatstage 3, where each certificate can be uploaded and have its URI recorded. With the URIs known, the attendee record can be updated and uploaded to MailChimp. Since the certificates were created automatically, stage 4 becomes redundant. This process is sumarised in figure \ref{fig:new-solution}. With such a script, CIET can be confident in its ability to deliver the correct certificates to its members on time and without significant human time investments.

\begin{figure}[b!]
    \centering
    \includegraphics[width=0.8\textwidth]{figures/new_solution.pdf}
    \caption{New solution flowchart.}
    \label{fig:new-solution}
\end{figure}

Realising this goal, however, will require a well organised team and a plan. Due to the various actions this script must perform, a moderately large codebase utilising several libraries and protocols is needed, demanding a significant time investment and expertise. This complexity is compounded by delegating the workload between a team, needing appropriate version control, task management, and enforeable coding standards. Furthermore, such a project could easily suffer from scope creep, especially with the user interface. Hence, the need for subsequent sections in this document to eliminate these issues.

\subsection{Prototype}

For the sake of time, a prototype can be implemented before scope creep becomes an issue. A prototype must include:

\begin{enumerate}
    \item Command line interface to run program.
    \item Must start program with certificate template path, attendee record path, and Mailchimp API and server keys as command line arguments
    \begin{enumerate}
        \item Strongly recommended to use \mintinline{python}{argparse} library.
    \end{enumerate}
    \item Generate a certificate of attendance for each attendee
    \begin{enumerate}
        \item Must be saved into a folder called \mintinline{console}{/certificates} in the current working directory.
        \item Each certificate must be named based on attendee's full name.
    \end{enumerate}
    \item Upload certificates to Mailchimp, keeping track of the file URIs.
    \begin{enumerate}
        \item Uploading of certificates must be done as a batch, not individually uploaded.
    \end{enumerate}
    \item Update the attendee record with the certificate URIs.
    \item Update the Mailchimp contacts with file URIs.
    \item Inform the user of what the program is currently doing.
    \begin{enumerate}
        \item Can just be a simple sentence, e.g. "Uploading certificates to Mailchimp..."
    \end{enumerate}
    \item Be able to process errors without crashing.
    \item Must be able to continue processing certificates when one fails.
    \item Must inform user of which step or certificates failed and why.
    \begin{enumerate}
        \item Can be done once the program has finished everything else.
    \end{enumerate}
    \item Must be able to run on Windows, Linux, and Mac systems capable of running Python 3.
\end{enumerate}

\noindent
For the sake of time and simplicity, the following assumptions can be made:

\begin{enumerate}
    \item The certificate of attendance is known.
    \begin{enumerate}
        \item The only field to be changed is the attendee name.
        \item The exact location of the attendee name field is known and constant.
        \item The placeholder text of the attendee name field is known and constant.
        \item The certificate is always valid.
    \end{enumerate}
    \item The attendee record contains no errors.
\end{enumerate}

\subsection{Minimal Viable Product}

Once a prototype is complete, a minimal viable product (MVP) can be written. On top of the criteria for the prototytpe, the MVP also has:

\begin{enumerate}
    \item Can be run via an executable.
    \begin{enumerate}
        \item No command line interfacing should be necessary.
    \end{enumerate}
    \item Should have a simple and initiative GUI.
    \item Can select input files via file explorer.
    \item Must log at least info, warning, error, and critical messages to a log file.
    \begin{enumerate}
        \item Debug messages are optional.
        \item It must be known where the log file is.
        \item The GUI should have a button for viewing the log file.
        \item Strongly recommended to use the \mintinline{python}{logging} library.
    \end{enumerate}
    \item Must use OAuth 2 to get Mailchimp authorisation.
    \begin{enumerate}
        \item The program should not have the option to use API keys.
    \end{enumerate}
    \item Should be able to inform the user of errors that occur in real time without interrupting background tasks.
    \item Use appropriate colour coding and symbols to inform user of successes, warnings, and errors.
    \item Must be able to instantly respond to user inputs without stuttering.
    \item Where ever possible, multiprocessing and multithreading must be used.
    \item Must be able to check attendee record for errors.
    \begin{enumerate}
        \item Repeated email addresses.
        \item Missing names or emails.
    \end{enumerate}
    \item Must be able to edit PDF document in GUI.
    \begin{enumerate}
        \item Must be able to view PDF document within GUI.
        \item Must be able to select the field where the full name goes.
    \end{enumerate}
\end{enumerate}

\subsection{Professional Product}

If the MVP is done, a professional, market ready, version can be done. This version should have everything the MVP has, and

\begin{enumerate}
    \item Has ultra-specific error messages.
    \begin{enumerate}
        \item Informs the user of what exaclty went wrong.
        \item Where possible, offers a solution.
        \item Where possible, offers links to online documentation.
    \end{enumerate}
    \item Must include install wizard with a build machine.
    \begin{enumerate}
        \item Must be able to choose where to install program.
    \end{enumerate}
    \item Allows user to select where log files go.
    \item Allows user to select what types of messages get logged.
    \item Must be able to select multiple fields in template as placeholders.
    \begin{enumerate}
        \item Must be able to select what each placeholder gets replaced with.
        \item Must be able to select between attendee attribute, constant value (e.g. event name), or system value (e.g. timestamp).
    \end{enumerate}
    \item Must be able to generate PDF report of what happened.
    \begin{enumerate}
        \item Can select where report gets saved to.
        \item Can select what goes on the report.
    \end{enumerate}
\end{enumerate}

\newpage

\section{Prerequisite Knowledge}

\subsection{Python Language}

While Python was originally designed as a simple scripting language, it has since evolved into one of the most used and popular languages. The simplicity of Python lends itself well to rapid prototyping, so simple that Python's syntax is often considered to be pseudocode. Furthermore, the library support for Python, both first and third-party, makes it extremely easy to make complex tasks simple. The sheer abundance of libaries available creates a compunding effect, as existing libraries makes it easier to wrote other libraries, leading to an exponential growth in library support. Whenever a new protocol or standard becomes prominent, chances are someone has already made a Python library to make it trivial to implement the protocol or standard within other programs. Additionally, the popularity of Python means there is ambundant support online for both known issues and asking online for assistance. For the sake of prototyping, it is difficult to find a better language than Python.

Considering the ambundance of support for Python, learning the language is best done with interactive tutorials foudn online. Such examples include W3 Schools, Codecademny, and Learn Python. Numerous other tutorials exist, such as those on Tutorials Point and the countless YouTube videos; however, these tutorials lack an interactive component, leading to reading or watching without learning. When learning a new language, the source is less important than the interaction between the material and the student.

Where Python's weakness lies is in its performance. Since Python is a weak typed language, datatype checking must be done during runtime, greatly reducing performance. Far worse, the interpreted nature of Python means compiler optimisations cannot be performed, slowing down Python to a crawl. Compared to optimised compiled languages, such as C, Rust, and C++, Python performs dozens of times worse when comparing equivelant programs. This can be greatly mitigated by taking advantage of scientific libraries, such as NumPy and Pandas. These libraries are written in C and C++, allowing for speeds within the ballpark of C, C++, and Rust. However, certain conventions must be followed to take advantage of vectorised data, leading to data needing to be structured in vectors or matrices. Unfortunately, this will only result in a performance increase when doing mathematical operations on numeric data, such as vector dot products or summing an array. Ergo, Python is a poor choice for performance critical applications.


\subsection{OSI Model}



\subsection{TCP/IP Model}



\subsection{HTTP Requests and Responses}

Of interest to interfacing with internet-based servers is the hypertext transfer protocol (HTTP).

\subsection{Object Orientation}



\subsection{Application Programming Interface}



\newpage

\section{PDF Editing and Creation}



\newpage

\section{MailChimp Authorisation}

\subsection{Using API and Server Keys}




\subsection{Using OAuth 2.0}



\newpage

\section{Certificate Uploading}




\newpage

\section{Contact Updating}



\newpage

\section{User Interface}

\subsection{Command Line Interface}






\subsection{Desktop GUI}




\newpage

\section{Version Control}

\subsection{What is Git?}



\subsection{Using Github}



\subsection{Recording Library Requirements}



\subsection{Virtual Environments}



\newpage

\section{Coding Standards}

\subsection{Readability and Maintainability}



\subsection{Documentating using Sphinx}



\newpage

%%\printbibliography

\end{document}